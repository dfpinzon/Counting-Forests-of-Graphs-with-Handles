\documentclass[xcolor=dvipsnames]{beamer}
%math symbols language
\usepackage{graphicx,amsmath,latexsym,amssymb,mathalfa}
%tables
\usepackage{booktabs}
%pictures and drawings
\usepackage{tikz}
%dots for matrices
\usepackage{mathdots}
%pdf maker
%\usepackage[plainpages=false,pdfpagelabels,colorlinks=false,urlcolor=black,pdfpagemode=UseNone,pdfstartview=FitH]{hyperref}
%pdf hyperlink refs
\usepackage{cleveref}
\usepackage{beamertexpower}
%cursive font
\usepackage[bb=boondox,bbscaled=.95,cal=boondoxo]{mathalfa}
%theorems style
\usepackage{amsthm}
%\usepackage[nodisplayskipstretch]{setspace}

%beamer presentation settings
\useoutertheme{infolines}
\usetheme[height=10mm]{Rochester}
\usefonttheme[onlymath]{serif}
\setbeamertemplate{items}[ball]
\setbeamertemplate{blocks}[rounded][shadow=true]
\setbeamertemplate{navigation symbols}{}
\usecolortheme[named=ForestGreen]{structure}
\author{\textsc{Patrick Graham }}
\title{\textsc Math 4500: Graph Theory}
%\institute{\textsc{GGC}}





\newenvironment{pf}
    {{\noindent  \textrm{\textbf{Proof }}}~~}

% MATH -------------------------------------------------------------------

\makeatletter
\renewcommand*\env@matrix[1][*\c@MaxMatrixCols c]{%
  \hskip -\arraycolsep
  \let\@ifnextchar\new@ifnextchar
  \array{#1}}
\makeatother


%fancy letters and typesettings
\newcommand{\nl}{\emptyset}
\newcommand{\p}{\partial}
\newcommand{\bR}{{\bf R}}
\newcommand{\bC}{{\bf C}}
\newcommand{\bZ}{{\bf Z}}
\newcommand{\bN}{{\bf N}}
\newcommand{\bQ}{{\bf Q}}
\newcommand{\bK}{{\bf K}}
\newcommand{\bI}{{\bf I}}
\newcommand{\bv}{{\bf v}}
\newcommand{\bV}{{\bf V}}
\newcommand{\cF}{{\mathcal F}}
\newcommand{\cA}{{\mathcal A}}
\newcommand{\cB}{{\mathcal B}}
\newcommand{\cC}{{\mathcal C}}
\newcommand{\cL}{{\mathcal L}}
\newcommand{\cP}{{\mathcal P}}
\newcommand{\cX}{{\mathcal X}}
\newcommand{\cY}{{\mathcal Y}}
\newcommand{\cZ}{{\mathcal Z}}
\newcommand{\bs}{\backslash}
\newcommand{\e}{{\epsilon}}
\newcommand{\gs}{{\sigma}}
\newcommand{\ve}{{\varepsilon}}
\newcommand{\ra}{\rightarrow}
\newcommand{\C}{{\mathbb C}}
\newcommand{\Q}{{\mathbb Q}}
\newcommand{\R}{{\mathbb R}}
\newcommand{\Z}{{\mathbb Z}}
\numberwithin{equation}{section} \DeclareMathOperator{\Var}{Var}
\DeclareMathOperator{\Ima}{Im}
\DeclareMathOperator{\sgn}{sgn}


\newcommand{\B}[1]{\textbf{#1}}
\newcommand{\tn}[1]{\textnormal{#1}}


%shortcuts
\newcommand{\abs}[1]{\left\vert#1\right\vert}
 \newcommand{\set}[1]{\left\{#1\right\}}
\newcommand{\ord}[1]{\left(#1\right)}
 \newcommand{\vct}[1]{\left<#1\right>}
 \newcommand{\norm}[1]{\left\Vert#1\right\Vert}
 \newcommand{\essnorm}[1]{\norm{#1}_{\ess}}
\newcommand{\N}{{\mathbb{N}}}
\newcommand{\UU}{\widetilde{U}(F)}
\newcommand{\bfrac}[2]{\displaystyle \frac{#1}{#2}}
\newcommand{\ds}[1]{\displaystyle {#1}}
\newcommand{\dsum}[2]{\ds{\sum_{#1}^{#2}} }
\newcommand{\ora}[1]{\overrightarrow{#1}}

\newcommand{\bpf}{\begin{proof}}
\newcommand{\epf}{\end{proof}}
\newtheorem{prop}{Proposition}[section]
\newcommand{\bpr}{\begin{prop}}
\newcommand{\epr}{\end{prop}}
\newcommand{\bdf}{\begin{definition}}
\newcommand{\edf}{\end{definition}}
\newcommand{\blm}{\begin{lemma}}
\newcommand{\elm}{\end{lemma}}
\newcommand{\bex}{\begin{example}\rm }
\newcommand{\eex}{\end{example}}
\newcommand{\bcor}{\begin{corollary}}
\newcommand{\ecor}{\end{corollary}}
\newcommand{\bthm}{\begin{theorem}}
\newcommand{\ethm}{\end{theorem}}
\newcommand{\be}{\begin{enumerate}}
\newcommand{\ee}{\end{enumerate}}
\newcommand{\bq}{\begin{equation}}
\newcommand{\eq}{\end{equation}}
\newcommand{\bb}{\begin{itemize}}
\newcommand{\eb}{\end{itemize}}
\newcommand{\bpw}{\begin{cases}}
\newcommand{\epw}{\end{cases}}

\hypersetup{pdfpagemode=UseNone}


% -------------------------------------------------------------------
\begin{document}
%%%%%%%%%%%%%%%Title frame
\begin{frame}[shrink=3]
  \frametitle{\textsc{\normalsize Patrick Graham \\ \hspace{6cm} Georgia Gwinnett College}}


\bigskip
\bigskip\bigskip\bigskip

{\centerline{\Huge\color{black} Graph Theory}}
\bigskip\bigskip
{\centerline{\Large \textsc{Chapter 1: Graphs}}}


%\bigskip
\bigskip\bigskip\bigskip

{\centerline{\color{black} \textsc{Section 1.1 Introductions and Definitions}}}

\smallskip

{\centerline{\color{black}\textsc{\today}}}

\end{frame}

%%%%%%%%%%%%%%%Frame 2

\begin{frame}{Introduction}
Why study graph theory? \pause We already know about so many different kinds:  \pause algebraic,  \pause logarithmic, \pause trigonometric,  \pause and so many more! 

\bigskip{}

However, like so many other domains of mathematics are wont to incorporate are the ideas of abstraction (Which allows mathematicians to approach a lot more cases at once) and applications (the different domains that can be affected by this study).

\bigskip{}

\pause Just a few of these applications are Computer Science, Linguistics, Physics and Chemistry, Biology, and more. 


\end{frame}

%%%%%%%%%%%%%%%Frame 

\begin{frame}{Graph with one handle rooted at 1}

\begin{figure}
    \centering
    \includegraphics[width=\linewidth]{Math 4500 Research- Spanning Trees with added handles/graph theory research 2019-12-17.pdf}
 
\end{figure}
\(T(G)\) 

\end{frame}

%%%%%%%%%%%%%%%Frame 

\begin{frame}{Graph with one handle rooted at 1}

\(T_{(1)(2)}(G)\)

\end{frame}

%%%%%%%%%%%%%%%Frame 

\begin{frame}{Graph with one handle rooted at 1}

\(T(G)+T_{(1)(2)}(G)\)

\end{frame}

%%%%%%%%%%%%%%%Frame 

\begin{frame}{Graph with one handle rooted at 2}
\(T(G)\)
\end{frame}

%%%%%%%%%%%%%%%Frame 

\begin{frame}{Graph with one handle rooted at 3}

\end{frame}

%%%%%%%%%%%%%%%Frame 

\begin{frame}{Graph with two handles rooted at 1 }

\end{frame}

%%%%%%%%%%%%%%%Frame 

\begin{frame}{Graph with two handles rooted at 2 }

\end{frame}

%%%%%%%%%%%%%%%Frame 

\begin{frame}{Graph with two handles rooted at 3}
\end{frame}

%%%%%%%%%%%%%%%Frame 

\begin{frame}{Graph with two handles rooted at 4}

\end{frame}

%%%%%%%%%%%%%%%Frame 

\begin{frame}{Graph with two handles rooted at 5}

\end{frame}

%%%%%%%%%%%%%%%Frame 

\begin{frame}{Graph with 3 handles rooted at 1}

\end{frame}

%%%%%%%%%%%%%%%Frame 
\begin{frame}{Graph with 3 handles rooted at 1}

\end{frame}

%%%%%%%%%%%%%%%Frame 
\begin{frame}{Graph with 3 handles rooted at 2}

\end{frame}

%%%%%%%%%%%%%%%Frame 
\begin{frame}{Graph with 3 handles rooted at 3}

\end{frame}

%%%%%%%%%%%%%%%Frame 
\begin{frame}{Graph with 3 handles rooted at 4}

\end{frame}

%%%%%%%%%%%%%%%Frame 
\begin{frame}{Graph with 3 handles rooted at 5}

\end{frame}

%%%%%%%%%%%%%%%Frame 
\begin{frame}{Graph with 3 handles rooted at 6}

\end{frame}

%%%%%%%%%%%%%%%Frame 
\begin{frame}{Graph with 3 handles rooted at 7}

\end{frame}

\end{document}
