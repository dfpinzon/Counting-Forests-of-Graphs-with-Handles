\documentclass[11pt]{article}
\usepackage{3500Quiz}

\begin{document}

\quizheading{1}{Patrick Graham}

\affirmation

\exercise{1.  Let \(a\in \Z.\)  Prove that \(-1\cdot a\) is the additive inverse of \(a\).}

\proof {This will be a direct proof. The additive inverse of a is an element (-a) such that \(a+(-a)=0\). Well the additive inverse of 1 is (-1) such that \(1 + (-1) = 0\). Well if we multiply both sides of the equality by a we have that \(a(1 + (-1)) = a\cdot0\). Well by the Distributive Law, We have that \(a\cdot1 + a\cdot(-1) = a\cdot0)\). Well by the proof of \(a\cdot0 = 0\) in a previous assignment we have that \(a\cdot1 + a\cdot(-1) = 0)\). Since \(a+(-a)=0\) and \(0=a\cdot1 + a\cdot(-1)\) We can directly equate these statements. By 1 being the multiplicative identity we have that \(a\cdot1 = a\) which means that (\(a+(-a) = a + a\cdot(-1)\) if we add the additive inverse of a to both sides we have the resulting equality \((-a) = -1\cdot a\). }

\qed

\exercise{2.  Finish the proof that the PMI and the WOP are equivalent by proving that the WOP implies the PMI.}

\proof{The Well-Ordering Principle (WOP) states that every nonempty set of the nonnegative integers has a smallest element. Assume that  the (WOP) holds and that S is a subset of nonnegative integers.  }

\qed

\exercise{3.  Prove that, if \(\gcd(a,b)=d,\) then \(\displaystyle \gcd\left(\frac{a}{d},\frac{b}{d}\right)=1.\)}

\proof {}

\qed

\exercise{4.  The \textbf{least common multiple} of nonzero integers \(a\) and \(b\) is the smallest positive integer \(m\) such that \(a\mid m\) and \(b\mid m.\)  It is denoted \(\lcm[a,b],\)  or sometimes \([a,b]\) for short.  Prove the following:
\begin{alphenum}
\item   If \(a\mid k\) and \(b\mid k,\) then \(\lcm[a,b]\mid k.\)
{}
\item   If \(\gcd(a,b)=1,\) then \(\lcm[a,b]=ab.\) {}
\item   If \(c>0,\) then \(\lcm[ca,cb]=c\cdot \lcm[a,b].\) {}
\item   If \(a>0\) and \(b>0,\) then \(\displaystyle \lcm[a,b]=\frac{ab}{\gcd(a,b)}.\) (This is a generalization of (b), and one of the most useful results in elementary number theory.)
\end{alphenum}
}

\proof {}

\qed

\end{document} 