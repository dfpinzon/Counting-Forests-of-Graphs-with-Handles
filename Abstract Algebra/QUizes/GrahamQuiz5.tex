\documentclass[11pt]{article}
\usepackage{3500Quiz}
\usepackage{bm}
\newcommand{\ol}[1]{\overline{#1}}
\newcommand{\vp}{\varphi}

\begin{document}

\quizheading{5}{Patrick Graham}

\affirmation

\exercise{1. Let \(R\) be a ring, and let \(I\) and \(J\) be ideals of \(R\).  Define the \textbf{sum} of \(I\) and \(J\) by \(I+J=\{a+b \; | \; a\in I, b\in J\}\).  Prove that \(I+J\) is an ideal of \(R\).
}

\proof
{
  First it must be shown that \(I+J\) is a sub-ring of \(R\). We will do this by relying on the condensed sub-ring test.
  \begin{itemize}
      \item[(i)] Let \((i_1 +j_1)\),\((i_2+j_2) \in I+J\). This means that \((i_1 +j_1)\) - \((i_2+j_2) = i_1 - i_2 + j_1 + j_2 = (i_1 - i_2) + (j_1 - j_2)\).
      I and J are both ideals, thus \(i_1 - i_2 \in I\) and \(j_1 - j_2 \in J\), thus by definition of \((i_1 - i_2) + (j_1 - j_2) \in I+J\).
      \item[(ii)] Let \((i_1 +j_1)\),\((i_2+j_2) \in I+J\). This means that \((i_1 +j_1)\)\((i_2+j_2) = i_1(i_2+j_2)+ j_1(i_2+j_2)\). We know that \(i_2+j_2 \in R\)  and that \(i_1 \in I\) hence \(i_1(i_2+j_2) \in I\) by definition of \(I\). A similar argument can be used for \(j_1(i_2+j_2) \in J\), Hence, \(i_1(i_2+j_2)+ j_1(i_2+j_2) \in I+J\).
  \end{itemize}
  thus \(I+J\) is a sub-ring of \(R.\) Now it must be shown that \(I + J\) is an Ideal of \(R\). Let \(r\in R\) and \(i+j \in I+J\) then
  \begin{itemize}
      \item[(i)]\(r(i+j) \in I\). Well, \(r(i+j) = ri + rj\). This means that \(ri \in I\) and \(rj \in J\), hence \(ri + rj \in I+J\).
      \item[(ii)]\((i+j)r \in I\). Well, \((i+j)r = ir + jr\). This means that \(ir \in I\) and \(jr \in J\) thus \(ir + jr \in I+J\).
  \end{itemize}
  Hence, \(I +J\) is a ideal of \(R\).
}

\qed

\exercise{2. Let \(R\) be a commutative ring with identity, and assume that \(\{0\}\) and \(R\) are the only ideals of \(R\).  Prove that \(R\) is a field.}

\proof
{
Let \(r \in R\) and r be nonzero. Let us look at the principle ideal generated by r written as (r). Since r is nonzero (r) \(\neq  \{0\}\), hence (r) \(= rR = R\) because \(R\) is the only other ideal. R has identity, this means that \(1 \in R\) and that \(1 \in (r)\). Any \(x \in (r)\) can be written as \(x= r \cdot r_1\) where \(r_1 \in R\) hence \(1= r\cdot x\). This shows that for all \(r\in R\) the equation \(1= rx\) has a solution.
}
\qed

\exercise{3. Let \(R\) be a ring, and set \(I:=\{(r,0) \; | \; r\in R\}\).  Prove that \(I\) is an ideal of \(R\times R\), and that \((R\times R)/I\cong R\).
}

\proof
{By Theorem 4.2.5, If \(\phi : R \rightarrow S\) is a homomorphism then ker \(\phi\) is an ideal of R. We must now find a homomorphism from \((R \times R) \rightarrow R\). Let the function \(f((r_1,r_2)) = r_2\) we will now show that this is a homomorphism.
\begin{itemize}
    \item[(i)] let \((r_1,r_2), (s_1,s_2) \in (R \times R)\). Then \(f(((r_1,r_2) + ((s_1,s_2)))) = f((r_1 + s_1,r_2 + s_2)) = r_2+s_2 = f((r_1,r_2))+f((s_1,s_2))\).
    \item[(ii)] let \((r_1,r_2), (s_1,s_2) \in (R \times R)\). Then \(f(((r_1,r_2)\cdot (s_1,s_2))= f((r_1 \cdot s_1,r_2 \cdot s_2)) = r_2 \cdot s_2 = f((r_1,r_2) \cdot f(s_1,s_2))\).
\end{itemize}}
Hence \(f(x,y)\) is a homomorphism and \(I\) is the kernal of \(R \times R\). Thus by theorem 4.2.5, \(I\) is an ideal. By the First Isomorphism Theorem, Domain/Kernal \(\cong\) Image. Hence \((R \times R)/I \cong R\) because \(f(r_1,r_2) = r_2 \in R\) image equals the co-domain.
\qed

\exercise{4. Prove that every ideal in \(\Z\) is principal.  First, show that any ideal \(I\) must contain positive integers.  By the WOP, \(I\) must contain a least element \(a\).  Finally, prove that \(I=(a)\).}

\proof
{Let I be an Ideal of \(\Z\), there are two types of ideals that can be considered, I equals the trivial ideal and thus is principle, or I is an ideal generated by nonzero integers. So let \(a\) be a nonzero integer, thus the ideal generated by \(a\) equals \(a\Z = \{a\cdot 0, a\cdot \pm 1, a\cdot \pm 2,...\}\). Clearly, I contains positive integers, and by the WOP,which states, Every nonempty set of the nonnegative integers has a smallest element, there must be a least element \(a\). Well by definition \(I = \{az | z \in Z\}\) and because the ring \(\Z\) is commutative \((a)  = a\Z = {az|z\in Z}.\) Thus \(I = (a).\) Hence every ideal in \(\Z\) is principle.}
\qed

\begin{solution}
All's well except for the last one, where you miss some important steps.

Total Points:  10.8/12
\end{solution}

\end{document} 