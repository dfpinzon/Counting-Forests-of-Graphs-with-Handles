\documentclass[11pt]{article}
\usepackage{3500Quiz}
\usepackage{bm}
\newcommand{\ol}[1]{\overline{#1}}
\newcommand{\vp}{\varphi}

\begin{document}

\quizheading{4}{Patrick Graham}

\affirmation

\exercise{1. Prove that, if \(\bm{\vp:\Z\rightarrow \Z}\) is an isomorphism, then \(\bm{\vp}\) is the identity map. }

\proof {Assume that \(\bm{\vp:\Z\rightarrow \Z}\) is an isomorphism, this means that
\begin{itemize}
\item[(i)] \(\bm{\vp}\) is bijective
\item[(ii)] \(\bm{\vp(r+s)}\) = \(\bm{\vp(r)} + \bm{\vp(s)}\)
\item[(iii)] \(\bm{\vp(rs)}\) = \(\bm{\vp(r)} \bm{\vp(s)}\)
\end{itemize}

We need to show is that \(\bm{\vp(r)} = r\), Because the ring \(\Z\) has identity and due to property (i) of isomorphisms \(\bm{\vp}\) is also surjective, due to property (v) of theorem 3.3.9 \(\bm{\vp(1)} = 1\). Well \(\bm{\vp(r)}= \bm{\vp(\underbrace{1+..+1}_{\text{r copies}})} =\underbrace{\bm{\vp(1)}+...+ \bm{\vp(1)}}_\text{r copies} = \underbrace{1 + ... + 1}_\text{r copies} = r.\)

}
\qed

\begin{solution}
Now I see the idea you hit upon that day in my office.  Great!

Points:  3
\end{solution}

\exercise{2. It can be shown that \(S:=\{\ol{0},\ol{4},\ol{8},\ol{12},\ol{16},\ol{20},\ol{24}\}\) is a subring of \(\Z_{28}\).  Prove that the map \(\psi:\Z_7\rightarrow S\) given by \(\psi([x])=\ol{8x}\) is a well-defined isomorphism.}

\proof
{
\begin{itemize}
    \item[(a)]bijective- The map \(\psi:\Z_7 \rightarrow S\) is a bijection because each element of the codomain is mapped to by exactly one element of the domain demonstrated below.
    \begin{itemize}
        \item[(i)] - \(\psi(\ol{0}) =  \ol{8\cdot0} =\ol{0}\)
        \item[(ii)] - \(\psi(\ol{1}) = \ol{8\cdot1} =\ol{8}\)
        \item[(iii)] - \(\psi(\ol{2}) = \ol{8\cdot2} =\ol{16}\)
        \item[(iiii)] - \(\psi(\ol{3}) = \ol{8\cdot3} =\ol{24}\)
        \item[(v)] - \(\psi(\ol{4}) = \ol{8\cdot4} =\ol{32} = \ol{4}\)
        \item[(vi)] -\(\psi(\ol{5}) = \ol{8\cdot5} =\ol{40} = \ol{12}\)
        \item[(vii)] - \(\psi(\ol{6}) = \ol{8\cdot6} =\ol{48} = \ol{20}\)

    \end{itemize}


    \item[(b)] perseves addition - \(\Z_7 = \{\ol{0},\ol{1},\ol{2},\ol{3},\ol{4},\ol{5},\ol{6}\} \psi([x]+[y]) = \psi([x]) + \psi([y]), \ol{8\cdot[x+y]} =\ol{8x+8y}= \ol{8x} + \ol{8y}\)
    \item[(c)]preserves multiplication - \(\psi([r][s])= \ol{8xy}. \) S is a sub-ring of \(\Z_{28}\) so  \(\ol{8} = \ol{64}\) because of \(64 \equiv 8 \mod{28}\). Thus \(\ol{8xy} = \ol{64xy} = \ol{8x8y} = \ol{8x}\ol{8y} = \psi(x)\psi(y).\)
    \item[(d)] well defined-Let \(\ol{x}\) and \(\ol{x\prime}\) be two representatives of the same congruence class of \(\Z_7\), then \(\ol{x}\) and \(\ol{x\prime}\) can be written as \(\ol{x} = 7n + a \) and \(\ol{x\prime} = 7m + a \), for \(n,m \in Z\) and \(a\in \{0,1,2,3,4,5,6\}\).Well \(\psi(\ol{x}) = \ol{8x} = \ol{8(7n+a)} = \ol{56n+8a}\) and \(\psi(\ol{x\prime}) = \ol{8x\prime} = \ol{8(7m+a)} = 56m+8a\) hence \(\psi(x) - \psi(x\prime) = 56n+8a -(56m+8a) = 56(n-m).\) Well \(28 \mid 56(m-n)\) so \(x\) and \(x\prime\) are congruent mod 28 thus the function is well defined.

\end{itemize}
}
\qed

\begin{solution}
This is another good proof.  It's thorough and clear.  Have you been paying more attention to grammar?  Keep it up!

Points:  3
\end{solution}

\exercise{3. Let \(R\) be a ring with identity.  If there is a smallest positive integer \(n\) such that \(n\cdot 1=0,\) then \(R\) is said to have \textbf{characteristic n}.  If no such \(n\) exists, then \(R\) is said to have \textbf{characteristic 0}.
\begin{itemize}
\item[(a)]  Let \(S\) and \(T\) be rings, and let \(f:S\rightarrow T\) be an isomorphism.  Prove that for any integer \(m>0\) and any \(r\in S\), we have \(f(mr)=mf(r)\).
\item[(b)]  Let \(R\) be a ring with identity and with characteristic \(n>0.\) Prove that any ring isomorphism on \(R\) preserves its characteristic.
\end{itemize}
}


\proof
{
\begin{itemize}
    \item[(a)]- f is an isomorphism hence \(f(mr)= f(\underbrace{r+..+r}_\text{m copies}) = \underbrace{f(r)+..+f(r)}_\text{m copies} = mf(r).\)
    \item[(b)]- Let \(\vp\) be any ring isomorphism. It follows that \(\vp(n\cdot1) = \vp(n)\vp(1) = \vp(n)\cdot1 = n \cdot 1 = 0 = \vp(0).\)
\end{itemize}
}
\qed

\begin{solution}
The first part is fine:  You build on your work in \#1.  For the second part, we must be a little more careful.  It's not true that \(\vp(n\cdot1) = \vp(n)\vp(1)\); only that \(\vp(n\cdot 1)=n\cdot \vp(1)=n\cdot 1=n\).  Perhaps this is what you meant.  In any case, it still remains to prove that \(n\) is the \emph{smallest} positive integer such that \(n\cdot 1_S=0\).  See the key for details.

Points:  2.7
\end{solution}

\exercise
{
4. Prove that isomorphism is an equivalence relation on the set of all rings.  You may assume elementary results regarding the inverses and compositions of bijections.
}

\proof
{ let \(\vp:R\rightarrow S \) be the isomorphism from a Ring to a Ring.
Well to show that it is an equivalence relation we need to show that \(\vp\) is reflexive, symmetric ,transitive.

\begin{itemize}
    \item[reflexive]-By Example 3.3.4 we saw that R \(\cong\) R is isomorphic for any ring.
    \item[symmetric]- Assume \(R\cong S \) then there exists a \(\vp^(-1):R\prime \rightarrow R\) because \(\vp\) is a bijection then \(\vp^(-1)\) is a bijection.  \
\end{itemize}

}
\qed

\begin{solution}
Ahh, now for this we require more detail.  Still, this quiz, overall, may be your best work yet.  Thank you also for your good typesetting.

Points:  1.8

Total Points:  10.5/12
\end{solution}

\end{document} 